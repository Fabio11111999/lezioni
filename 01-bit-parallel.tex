\documentclass[12pt,aspectratio=169]{beamer}
\usepackage{pxfonts}

\usepackage{fancyvrb}
\fvset{%frame=single,
commandchars=\\\{\},
%framesep=1mm,
fontfamily=helvetica,
fontsize=\normalsize
}
\usepackage[bitstream-charter]{mathdesign}
\usepackage{listings}
\lstset{commentstyle=\color{orange}, keywordstyle=\color{yellow} }


\usetheme{Boadilla}
\useoutertheme{split}
\usecolortheme{albatross}


\setbeamertemplate{blocks}[rounded][shadow=false]
\setbeamertemplate{navigation symbols}{}

\setbeamercolor*{structure}{fg=green!75!black,bg=blue!70!white}
\setbeamercolor*{normal text}{fg=green!65!black,bg=blue!80!black}
\setbeamercolor{palette primary}{use={structure,normal text},fg=green,bg=structure.bg!75!black}
\setbeamercolor{palette secondary}{use={structure,normal text},fg=structure.fg,bg=structure.bg!60!black}
\setbeamercolor{palette tertiary}{use={structure,normal text},fg=structure.fg,bg=structure.bg!45!black}
\setbeamercolor{palette quaternary}{use={structure,normal text},fg=green,bg=structure.bg!75!black}
\setbeamercolor*{example text}{fg=green!65!black}
\setbeamercolor*{block body}{bg=structure.bg!90!black}
\setbeamercolor*{block body alerted}{bg=structure.bg!90!black}
\setbeamercolor*{block body example}{bg=structure.bg!90!black}
\setbeamercolor*{block title}{parent=structure,bg=structure.bg!75!black}
\setbeamercolor*{block title alerted}{use={structure,alerted text},fg=alerted text.fg!75!structure.fg,bg=structure.bg!75!black}
\setbeamertemplate{navigation symbols}{}
\setbeamertemplate{items}[square]
\setbeamercolor{item projected}{fg=white}
\setbeamercolor*{normal text}{fg=white!90!blue,bg=blue!70!black}
\setbeamercolor*{separation line}{}
\setbeamercolor*{fine separation line}{}
\setbeamercolor{alerted text}{fg=green}

\usepackage[italian]{babel}
\usepackage[utf8]{inputenc}
\usepackage{pgf}
\usepackage{verbatim}
\usepackage{inconsolata}
\usepackage{listings}
\lstset{language=C, frame=single,
  basicstyle=\ttfamily,
  numbers=left, numberstyle=\tiny\color{gray},
  numbersep=5pt, fancyvrb=true
}

\usefonttheme{professionalfonts} % using non standard fonts for beamer
\usefonttheme{serif} % default family is serif
% \usepackage{fontspec}
% \setmainfont{Palatino}

\usepackage{pgf}
\usepackage{tikz}
\usepackage{graphicx}
\usetikzlibrary{%
  arrows,
  arrows.meta,
  positioning,
  calc,
  backgrounds,
  chains,
  matrix,
  patterns,
  automata,
  fit,
  graphs,
  decorations,
  decorations.pathmorphing,
  decorations.pathreplacing,
  decorations.markings,
}

\usepgflibrary{shapes,shapes.geometric}


\usepackage{url}
\usepackage{xmpmulti}
% \usepackage{euler}
\usepackage[T1]{fontenc}
\pdfpagebox5
% \immediate\write18{sh ./vc}
% \input{vc}

\author{Gianluca Della Vedova}
\title{Elementi di Bioinformatica}
\institute{Univ. Milano--Bicocca\\
  \texttt{http://gianluca.dellavedova.org}}
\date{\today}
%\pgfdeclareimage[height=1cm]{university-logo}{logounimib}
%\logo{\pgfuseimage{university-logo}}


\begin{document}

\begin{frame}\frametitle{Gianluca Della Vedova}
\begin{itemize}
\item
Elementi di Bioinformatica
\item
Ufficio U14-2041
\item
\url{https://gianluca.dellavedova.org}
\item
\url{https://elearning.unimib.it/course/view.php?id=19214}
\item
\url{gianluca.dellavedova@unimib.it}
\item\url{https://github.com/bioinformatica-corso/programmi-elementi-bioinformatica}
\item\url{https://github.com/bioinformatica-corso/lezioni}
\end{itemize}
\end{frame}

\begin{frame}[fragile]
\frametitle{Notazione}
\begin{itemize}
\item
\alert{simbolo}: $T[i]$\\
\item
\alert{stringa}: $T[1]T[2]\cdots T[l]$\\
\item
\alert{sottostringa}: $T[i:j]$\\
\item
\alert{prefisso}: $T[:j]=T[1:j]$\\
\item
\alert{suffisso}: $T[i:]=T[i:|T|]$
\item
\alert{concatenazione}: $T_{1}\cdot T_{2} = T_{1}T_{2}$
\end{itemize}
\end{frame}


\begin{frame}[fragile]
\frametitle{Pattern Matching}
\begin{block}{Problema}
\alert{Input}: testo $T=T[1]\cdots T[n]$, pattern $P=P[1]\cdots P[m]$, alfabeto $\Sigma$\\
\alert{Goal}: trovare \emph{tutte} le occorrenze di $P$ in $T$\\
\alert{Goal}: trovare tutti gli $i$ tale che $T[i]\cdots T[i+m-1]=P$
\end{block}
\begin{block}{Algoritmo banale}
\alert{Tempo}: $O(nm)$
\end{block}
\begin{block}{Lower bound}
\alert{Tempo}: $O(n+m)$
\end{block}
\end{frame}

\begin{frame}
\frametitle{Bit-parallel}
\begin{block}{Algoritmi seminumerici}
\begin{itemize}
\item
$25$
\item
$25=00011001$
\item
$25=00011001=$FFFTTFFT
\end{itemize}
\end{block}
\begin{block}{Operazioni bit-level}
\alert{Or}: $x\lor y$, \alert{And}: $x\land y$, \alert{Xor}: $x\oplus y$\\
\alert{Left Shift}: $x << k$, \alert{Right Shift}: $x >> k$,
\begin{itemize}
\item
Tutte bitwise
\item
Tutte in hardware
\end{itemize}
\end{block}
\end{frame}

\begin{frame}
\frametitle{D\"om\"olki / Baeza-Yates, Gonnet}
\begin{block}{Matrice $M$}
$M(i,j)=1$ sse $P[:i]=T[j-i+1:j]$\\
$0\le i\le m$, $0\le j\le n$
\end{block}
\begin{block}{Occorrenza di $P$ in $T$}
$M(m,\cdot)=1$
\end{block}
\begin{itemize}
\item
$M(0,\cdot)=1$, $M(\cdot,0)=0$
\item
\alert{$M(i,j)=1$} sse $M(i-1, j-1)=1$ AND $P[i]=T[j]$
\end{itemize}
\end{frame}

\begin{frame}
\frametitle{Esempio}
\begin{block}{Esempio}
$T$=abracadabra\\
$P$=abr
\end{block}

\begin{center}
\begin{tabular}[l]{ll}
%\hline{1}
10010101001\\
01000000100\\
00100000010&$\leftarrow$ \alert{occorrenze}\\%\hline
\end{tabular}
\end{center}

\begin{block}{Matrice $M$}
1 colonna = 1 numero
\end{block}
\end{frame}

\begin{frame}[fragile]
\frametitle{Colonne}
$U[\sigma]$ = array di bit dove $U[\sigma,i]=1$ sse $P[i]=\sigma$

\begin{block}{$C[j]$ da $C[j-1]$}
\begin{itemize}
\item
Right shift di $C[j-1]$
\item
$1$ in prima posizione
\item
AND con $U[T[j]]$
\item
\alert{$\omega$}: word size
\item
$C[j] = \left( \left(C[j-1] >> 1 \right) \  | \  \left(1 << (\omega -1) \right) \right)\ \&\  U[T[j]]$;
\end{itemize}
\end{block}
\end{frame}

\begin{frame}[fragile]
\frametitle{Note}
\begin{itemize}
\item
Tempo $O(n)$ se $m\le \omega$
\item
Tempo $O(nm)$
\item
No condizioni
\item
$\omega < m\le 2\omega$?
\end{itemize}
\end{frame}


\begin{frame}[containsverbatim]\frametitle{Licenza d'uso}
  \small

  Quest'opera {\`e} soggetta alla licenza Creative Commons:
Attribuzione-Condividi allo stesso modo 4.0.
  (\verb+https://creativecommons.org/licenses/by-sa/4.0/+).

Sei libero di riprodurre, distribuire, comunicare al pubblico, esporre
in pubblico, rappresentare, eseguire, recitare e modificare quest'opera
alle seguenti condizioni:
\begin{itemize}
\item
Attribuzione — Devi attribuire la paternit{\`a} dell'opera nei modi
indicati dall'autore o da chi ti ha dato l'opera in licenza e in modo tale da
non suggerire che essi avallino te o il modo in cui tu usi l'opera.
\item
Condividi allo stesso modo — Se alteri o trasformi quest'opera, o se
la usi per crearne un'altra, puoi distribuire l'opera risultante solo con
una licenza identica o equivalente a  questa.
\end{itemize}
%  \vspace*{1cm}
\end{frame}

\end{document}
