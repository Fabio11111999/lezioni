\documentclass[12pt,aspectratio=169]{beamer}
\usepackage{pxfonts}

\usepackage{fancyvrb}
\fvset{%frame=single,
commandchars=\\\{\},
%framesep=1mm,
fontfamily=helvetica,
fontsize=\normalsize
}
\usepackage[bitstream-charter]{mathdesign}
\usepackage{listings}
\lstset{commentstyle=\color{orange}, keywordstyle=\color{yellow} }


\usetheme{Boadilla}
\useoutertheme{split}
\usecolortheme{albatross}


\setbeamertemplate{blocks}[rounded][shadow=false]
\setbeamertemplate{navigation symbols}{}

\setbeamercolor*{structure}{fg=green!75!black,bg=blue!70!white}
\setbeamercolor*{normal text}{fg=green!65!black,bg=blue!80!black}
\setbeamercolor{palette primary}{use={structure,normal text},fg=green,bg=structure.bg!75!black}
\setbeamercolor{palette secondary}{use={structure,normal text},fg=structure.fg,bg=structure.bg!60!black}
\setbeamercolor{palette tertiary}{use={structure,normal text},fg=structure.fg,bg=structure.bg!45!black}
\setbeamercolor{palette quaternary}{use={structure,normal text},fg=green,bg=structure.bg!75!black}
\setbeamercolor*{example text}{fg=green!65!black}
\setbeamercolor*{block body}{bg=structure.bg!90!black}
\setbeamercolor*{block body alerted}{bg=structure.bg!90!black}
\setbeamercolor*{block body example}{bg=structure.bg!90!black}
\setbeamercolor*{block title}{parent=structure,bg=structure.bg!75!black}
\setbeamercolor*{block title alerted}{use={structure,alerted text},fg=alerted text.fg!75!structure.fg,bg=structure.bg!75!black}
\setbeamertemplate{navigation symbols}{}
\setbeamertemplate{items}[square]
\setbeamercolor{item projected}{fg=white}
\setbeamercolor*{normal text}{fg=white!90!blue,bg=blue!70!black}
\setbeamercolor*{separation line}{}
\setbeamercolor*{fine separation line}{}
\setbeamercolor{alerted text}{fg=green}

\usepackage[italian]{babel}
\usepackage[utf8]{inputenc}
\usepackage{pgf}
\usepackage{verbatim}
\usepackage{inconsolata}
\usepackage{listings}
\lstset{language=C, frame=single,
  basicstyle=\ttfamily,
  numbers=left, numberstyle=\tiny\color{gray},
  numbersep=5pt, fancyvrb=true
}

\usefonttheme{professionalfonts} % using non standard fonts for beamer
\usefonttheme{serif} % default family is serif
% \usepackage{fontspec}
% \setmainfont{Palatino}

\usepackage{pgf}
\usepackage{tikz}
\usepackage{graphicx}
\usetikzlibrary{%
  arrows,
  arrows.meta,
  positioning,
  calc,
  backgrounds,
  chains,
  matrix,
  patterns,
  automata,
  fit,
  graphs,
  decorations,
  decorations.pathmorphing,
  decorations.pathreplacing,
  decorations.markings,
}

\usepgflibrary{shapes,shapes.geometric}


\usepackage{url}
\usepackage{xmpmulti}
% \usepackage{euler}
\usepackage[T1]{fontenc}
\pdfpagebox5
% \immediate\write18{sh ./vc}
% \input{vc}

\author{Gianluca Della Vedova}
\title{Elementi di Bioinformatica}
\institute{Univ. Milano--Bicocca\\
  \texttt{http://gianluca.dellavedova.org}}
\date{\today}
%\pgfdeclareimage[height=1cm]{university-logo}{logounimib}
%\logo{\pgfuseimage{university-logo}}


\begin{document}

\begin{frame}
  \titlepage
\end{frame}


\begin{frame}[fragile]
\frametitle{Karp-Rabin}
\begin{block}{Alfabeto binario}
\begin{itemize}
\item
$H(S)=\sum_{i=1}^{|S|} 2^{i-1}H(S[i])$
\item
sliding window di ampiezza $m$ su $T$
\item
$H(T[i+1:i+m]) =$\\
$=\left(H(T[i:i+m-1]) - T[i] \right) / 2 + 2^{m-1}T[i+m]$
\item
operazioni su bit
\item
$T[i:i+m-1]=P \Leftrightarrow H(T[i:i+m-1])=H(P)$
\end{itemize}
\end{block}
\end{frame}

\begin{frame}[fragile]
\frametitle{Karp-Rabin: problema}
\begin{block}{Numeri troppo grandi}
\begin{itemize}
\item
Modello RAM: numeri $O(n+m)$
\item
mod $p$
\item
$H(T[i+1:i+m]) =$\\
$\left(\left(H(T[i:i+m-1]) - T[i] \right) / 2 + 2^{m-1}T[i+m] \right)\mod p$
\item \textbf{NO}
\item
$2^{m-1}T[i+m] \mod p$ calcolato iterativamente, $\mod p$ ad ogni passo
\end{itemize}
\end{block}
\end{frame}

\begin{frame}[fragile]
\frametitle{Karp-Rabin: falsi positivi}
\begin{block}{Possibili errori}
\begin{itemize}
\item
Falso positivo (FP): occorrenza non vera
\item
Falso negativo (FN): occorrenza non trovata
\item
$H(T[i:i+m-1])=H(P) \Leftrightarrow T[i:i+m-1]=P$
\item
$H(T[i:i+m-1])  \mod p = H(P)  \mod p$
$\Leftarrow T[i:i+m-1]=P$
\end{itemize}
\end{block}
\end{frame}


\begin{frame}[fragile]
\frametitle{Karp-Rabin: falsi positivi}
\begin{block}{Probabilità di errore}
$P[\#FP\ge 1] \le O(nm/I)$ se il numero primo $p$ è scelto fra tutti i primi $\le
I$
\end{block}

\begin{block}{Valori di $I$}
\begin{itemize}
\item
$I=n^{2}m \Rightarrow P[\#FP\ge 1] \le 2.54/n$
\item
$I=nm^{2}  \Rightarrow P[\#FP\ge 1] \in O(1/m)$
\end{itemize}
\end{block}

\begin{block}{Abbassare probabilità di errore}
Scegliere $k$ primi casuali (indipendenti senza ripetizioni), cambiare primo
dopo ogni FP
\end{block}
\end{frame}

\begin{frame}[fragile]
\frametitle{Las Vegas vs.
  Monte Carlo}
\begin{block}{Classificazione algoritmi probabilistici}
\begin{itemize}
\item
Monte Carlo:
\begin{itemize}
\item
Sempre veloce

\item
Forse non corretto
\item
Karp-Rabin
\end{itemize}
\item
Las Vegas:
\begin{itemize}
\item
Sempre corretto
\item
Forse non veloce
\item
Quicksort con pivot random
\end{itemize}
\end{itemize}
\end{block}
\end{frame}


\begin{frame}[fragile]
\frametitle{Controllo falsi positivi}
$L$: posizioni iniziali in $T$ delle occorrenze
\begin{block}{Run}
sequenza $\langle l_{1}, \ldots, l_{k}\rangle$ di posizioni in $L$ distanti al
massimo $m/2$
\end{block}

\begin{itemize}
\item
$d=l_{2}-l_{1}$
\item
$P$ semiperiodico con periodo $d$
\item
$P=\alpha\beta^{k-1}$, $\alpha$ suffisso di $\beta$
\item
ogni run occupa $\ge n$ caratteri di $T$
\item
ogni carattere  di $T$ è in max $2$ run
\end{itemize}
\end{frame}



\begin{frame}[containsverbatim]\frametitle{Licenza d'uso}
  \small

  Quest'opera {\`e} soggetta alla licenza Creative Commons:
Attribuzione-Condividi allo stesso modo 4.0.
  (\verb+https://creativecommons.org/licenses/by-sa/4.0/+).

Sei libero di riprodurre, distribuire, comunicare al pubblico, esporre
in pubblico, rappresentare, eseguire, recitare e modificare quest'opera
alle seguenti condizioni:
\begin{itemize}
\item
Attribuzione — Devi attribuire la paternit{\`a} dell'opera nei modi
indicati dall'autore o da chi ti ha dato l'opera in licenza e in modo tale da
non suggerire che essi avallino te o il modo in cui tu usi l'opera.
\item
Condividi allo stesso modo — Se alteri o trasformi quest'opera, o se
la usi per crearne un'altra, puoi distribuire l'opera risultante solo con
una licenza identica o equivalente a  questa.
\end{itemize}
%  \vspace*{1cm}
\end{frame}

\end{document}
