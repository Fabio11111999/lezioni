\begin{frame}
  \titlepage
\end{frame}




\begin{frame}[fragile]
\frametitle{Pattern matching su suffix array}
\begin{block}{Occorrenza $P$ in $T$}
Suffissi di $T$ che iniziano con $P$
\end{block}
\begin{block}{Ricerca in $SA$}
\begin{itemize}
\item
Ricerca dicotomica
\item
Tempo $O(m \log n)$ -- caso pessimo
\item
Controllare tutto $P$ ad ogni iterazione
\item
$log_{2} n$ iterazioni
\end{itemize}
\end{block}
\end{frame}

\begin{frame}[fragile]
\frametitle{Accelerante 1}
\begin{columns}
\begin{column}{0.6\textwidth}
\begin{itemize}
\item
Intervallo $SA(L, R)$ di $SA$
\item
Elemento mediano $M$
\item
Tutti i suffissi in $SA(L,R)$ iniziano con uno stesso prefisso lungo $Lcp(SA[L],
SA[R])$
\item
Non confrontare con i primi $Lcp(SA[L], SA[R])$ caratteri
\end{itemize}
\end{column}
\begin{column}{0.35\textwidth}
\begin{center}
\onslide<1->{\multiinclude[<+>][start=1,format=pdf,graphics={width=0.95\textwidth}]{figures/SA-pattern-matching1}}
%\onslide<4->{\includegraphics[width=\textwidth]{figures/trie3}}
\end{center}
\end{column}
\end{columns}
\end{frame}

\begin{frame}[fragile]
\frametitle{Accelerante 2}
\begin{columns}[T]
\begin{column}[T]{0.6\textwidth}
\begin{block}{$l$: $lcp(L,P)$; $r$: $Lcp(R,P)$}
\begin{enumerate}
\item
Caso 1: $l>r$
\only<3->{\begin{itemize}\item
$Lcp(L,M)>l$\only<4->{$\Rightarrow L\gets M$}
\only<5->{\item
$Lcp(L,M)<l$\only<6->{$\Rightarrow$\\ $R\gets M, r\gets Lcp(M,L)$}}
\only<7->{\item
  $Lcp(L,M)=l$\only<8->{$\Rightarrow$ confronto $P[l+1:]$, $M[l+1:]$}
}\end{itemize}}
\only<9->{\item
  Caso 2: $l=r$
\only<10->{\begin{itemize}\item
  $Lcp(L,M)>l$
\only<11->{\item $Lcp(M,R)>l$}
\only<12->{\item $Lcp(L,M)=Lcp(M,R)=l$}
  \end{itemize}}}
\end{enumerate}
\end{block}
\end{column}
\begin{column}{0.32\textwidth}
\only<2->{\multiinclude[<+>][start=1,format=pdf,graphics={width=0.95\textwidth}]{figures/SA-accelerant}}
%\only<2->{\multiinclude[<alert@+| +->][start=1,format=pdf,graphics={width=0.95\textwidth}]{figures/SA-accelerant}}
\end{column}
\end{columns}
\end{frame}

\begin{frame}[fragile]
\frametitle{Accelerante 3: calcolo $Lcp$ in tempo $O(n)$}
\begin{itemize}[<+->]
\item
Iterazione 1: $(L,R)=(1,n)$
\item
Iterazione 2: $(L,R)=(1,n/2)$ oppure $(n/2,n)$
\item
Iterazione $k$: $L = h\frac{n}{2^{k-1}}$, $R = (h+1)\frac{n}{2^{k-1}}$
\item
Iterazione $\lceil \log_{2}n\rceil$: $R=L+1$, $Lcp(h,h+1)$
\item
Iterazione $\lceil \log_{2}n\rceil -1$: aggrego i risultati dell'iterazione
$\lceil \log_{2}n\rceil$
\item
Iterazione $k$: $Lcp(h\frac{n}{2^{k-1}}, (h+1)\frac{n}{2^{k-1}})$
\item
$t=\frac{n}{2^{k}}$,
$Lcp\left(2ht + 1, \left(2h+2\right)t\right))=\min\{$
$Lcp\left( 2ht + 1, \left( 2h+1 \right) t\right),$
$Lcp\left( \left(2h+1 \right)t +1, \left(2h+2\right)t\right),$
$Lcp\left( \left(2h+1\right)t,\left(2h+1\right)t+1\right)
\}$
\end{itemize}
\end{frame}

\begin{frame}[fragile]
\begin{columns}[T]
\begin{column}[T]{0.8\textwidth}
\frametitle{Accelerante 3: calcolo $Lcp$ in tempo $O(n)$}
\multiinclude[<+>][start=1,format=pdf,graphics={width=\textwidth}]{figures/SA-linear-Lcp}
\end{column}
\begin{column}{0.18\textwidth}
\uncover<5->{\vfill Passaggio da $s$ a $z$ deve esistere}
\end{column}
\end{columns}
\end{frame}

\begin{frame}
\frametitle{Osservazione}
\begin{itemize}[<+->]
\item
Tempo per trovare un'occorrenza
\item
Tempo per trovare tutte le occorrenze?
\item
$O(n+m+k)$, per $k$ occorrenze | scansione lineare a partire dall'occorrenza trovata
\end{itemize}
\end{frame}

\begin{frame}
\frametitle{Costruzione suffix array: nuovo alfabeto}
\begin{itemize}
\item
Alfabeto $\Sigma$ con $\sigma$ simboli, testo $T$ lungo $n$
\item
Aggrego triple di caratteri
\item
Alfabeto $\Sigma^{3}$ con $\sigma^{3}$ simboli, testo lungo $n/3$
\item
$T_{1}=(T[1],T[2],T[3])\cdots (T[3i+1],T[3i+2],T[3i+3])\cdots$\\
$T_{2}=(T[2],T[3],T[4])\cdots (T[3i+2],T[3i+3],T[3i+4])\cdots$\\
$T_{0}=(T[3],T[4],T[5])\cdots (T[3i],T[3i+1],T[3i+2])\cdots$
\item
suffissi$(T)$ $\Leftrightarrow$ $\bigcup_{i=0,1,2}$ suffissi$(T_{i})$
\end{itemize}
\end{frame}

\begin{frame}
\frametitle{Costruzione suffix array: ricorsione}
\begin{enumerate}
\item
Ricorsione su $T_{0}T_{1}$
\item
suffissi$(T_{0}T_{1})$ $\Leftrightarrow$ suffissi$(T_{0})$, suffissi$(T_{1})$
\item
suffissi$(T_{0}T_{1})$  $\Leftrightarrow$ suffissi$(T_{2})$
\item
$T_{2}[i:] \approx T[3i+2:]$
\item
$T[3i+2:] = T[3i+2]T[3i+3:] =T[3i+2]T_{0}[i+1:]$
\item
suffissi$(T_{0})$ ordinati
\item
Radix sort
\item
Fusione suffissi$(T_{0}T_{1})$,  suffissi$(T_{2})$
\end{enumerate}
\end{frame}

\begin{frame}
\frametitle{Costruzione suffix array: fusione}
Confronto suffisso di $T_{0}$ e $T_{2}$
\begin{enumerate}
\item
$T_{0}[i:] <=> T_{2}[j:]$
\item
$T[3i:] <=> T[3j+2:]$
\item
$T[3i]T[3i+1:] <=> T[3j+2]T[3j+3:]$
\item
$T[3i]T_{1}[i:] <=> T[3j+2]T_{0}[j+1:]$
\end{enumerate}
\end{frame}



\begin{frame}
\frametitle{Costruzione suffix array: fusione}
Confronto suffisso di $T_{1}$ e $T_{2}$
\begin{enumerate}
\item
$T_{1}[i:] <=> T_{2}[j:]$
\item
$T[3i+1:] <=> T[3j+2:]$
\item
$T[3i+1]T[3i+2:] <=> T[3j+2]T[3j+3:]$
\item
$T[3i+1]T[3i+2]T[3i+3:] <=> T[3j+2]T[3j+3]T[3j+4:]$
\item
$T[3i+1]T[3i+2]T_{0}[i+1:] <=> T[3j+2]T[3j+3]T_{1}[j+1:]$
\end{enumerate}
\end{frame}

\begin{frame}
\frametitle{KS}
\begin{enumerate}
\item
Juha Kärkkäinen, Peter Sanders and Stefan Burkhardt.
Linear work suffix array construction. J. ACM, 53 (6), 2006, pp. 918-936.
\item
Difference cover (DC) 3
\item
Stefan Burkhardt and Juha Kärkkäinen.
Fast lightweight suffix array construction and checking
In Proc. 14th Symposium on Combinatorial Pattern Matching (CPM '03), LNCS 2676,
Springer, 2003, pp. 55-69. \url{http://www.stefan-burkhardt.net/CODE/cpm_03.tar.gz}
\item
Yuta Mori.
SAIS \url{https://sites.google.com/site/yuta256/}
\end{enumerate}
\end{frame}


\begin{frame}[fragile]
\frametitle{Sottostringa comune più lunga}
\begin{block}{$k$ stringhe $\{s_{1}, \ldots , s_{k}\}$}
\begin{enumerate}[<+->]
\item
Suffix tree generalizzato
\item
Vettore $C_{x}[1:k]$ per ogni nodo $x$
\item
$C_{x}[i]$: sottoalbero con radice $x$ ha una foglia di $s_{i}$
\item
$C_{x} = \bigvee C$ sui figli di $C$
\item
Nodo $z$, $C_{z}=$ tutti ${1}$
\item
Tempo $O(kn)$
\item
$n$: summa lunghezze $|s_{1}| + \cdots + |s_{k}|$
\end{enumerate}
\end{block}
\end{frame}


\begin{frame}[fragile]
\frametitle{Lowest common ancestor (lca)}
\begin{block}{Dati albero $T$ e 2 foglie $x$, $y$}
\begin{itemize}
\item
$z$ è antenato comune di $x$, $y$ se
$z$ è antenato di entrambi $x$ e $y$
\item
$z$ è lca di $x$, $y$ se:
\begin{enumerate}
\item
$z$ è antenato comune di $x$ e $y$
\item
nessun discendente di $z$ è antenato comune di $x$ e $y$
\end{enumerate}
\end{itemize}
\end{block}
\begin{block}{Proprietà}
\begin{itemize}
\item
Preprocessing di $T$ in tempo $O(n)$
\item
Calcolo lca$(x,y)$ in tempo $O(1)$
\item
Algoritmo complesso, ma pratico
\end{itemize}
\end{block}
\end{frame}



\begin{frame}[fragile]
\frametitle{Sottostringa comune più lunga di $k$ stringhe}
\begin{block}{Arricchimento $ST$}
\begin{enumerate}[<+->]
\item
$N_{x}[i]$: numero foglie di $s_{i}$ discendenti di $x$
\item
$N_{x}[i]=0$ o $1$ per ogni foglia
\item
$N_{x}[i]=$ somma dei figli
\item
$D_{x}[i]$: numero di consecutive di foglie di $s_{i}$, ordinate secondo visita depth-first, discendenti di $x$
\item
$N_{x}[i]=0 \Rightarrow D_{x}[i]=0$
\item
$N_{x}[i]=1 \Rightarrow D_{x}[i]=0$
\item
$N_{x}[i]\ge 1 \Rightarrow D_{x}[i]=N_{x}[i]-1$
\item
$N_{x}[i] - D_{x}[i] =$ \uncover<2->{$C_{x}[i]$}
\end{enumerate}
\end{block}
\end{frame}

\begin{frame}[fragile]
\frametitle{Sottostringa comune più lunga di $k$ stringhe}
\begin{block}{Gestione $ST$}
\begin{itemize}
\item
Visita depth-first di $ST$
\item
$L_{i}$: lista ordinata delle foglie di $s_{i}$
\item
Per ogni coppia $x,y$ consecutiva in $L_{i}$
\begin{enumerate}
\item
$z\gets lca(x,y)$
\item
$D_{z}[i]=$
\item
Aggiorna $C_{z}$
\end{enumerate}
\end{itemize}
\end{block}
\end{frame}



%%% Local Variables:
%%% TeX-PDF-mode: t
%%% TeX-master: "04-pattern-matching-suffix-array-video"
%%% End:
